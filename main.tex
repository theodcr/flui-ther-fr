% a4paper is essential, fleqn for left-alignment of formulas, leqno for left-sided formula numbers
\documentclass[10pt,a4paper,twocolumn,fleqn]{article}

\usepackage[utf8]{inputenc}
\usepackage[T1]{fontenc}

% Fonts, better looking with microtype
% Math design for matching maths : for charter and utopia
% fourier and kpfonts are good packages too
\usepackage{lmodern}
\usepackage[charter]{mathdesign}
\usepackage{microtype}

% Language
\usepackage[french]{babel}

% Geometry
\usepackage[top=2.5cm, bottom=3cm, left=1.5cm, right=1.5cm, headheight=15pt]{geometry}

% Figures and tikz
\usepackage{graphicx}
\usepackage{tikz}

% Colors
\definecolor{red1}{RGB}{204,0,0}

% Math
\usepackage{amsmath}
%\usepackage{amssymb} 
%\usepackage{mathrsfs}  % unusable with mathdesign
\usepackage[overload]{empheq}
\newcommand{\abs}[1]{\left\lvert#1\right\rvert}
\newcommand{\norme}[1]{\left\lVert#1\right\rVert}

% More greek letters
\usepackage{upgreek}

% Nomenclature
\usepackage[french]{nomencl}
\makenomenclature

% Utf8 characters in listings
%\usepackage{listingsutf8}

% Links for sections and references
\usepackage[colorlinks,linkcolor=black,urlcolor=red1,citecolor=red1]{hyperref}

% Space between paragraphs
\addtolength{\parskip}{2pt}

% Figures path
\graphicspath{{figures/}}

% Changes the look of sections, titlerule to add a line
\usepackage{titlesec}
\titleformat{\section}{\Large}{\thesection.}{.5em}{}[\titlerule]
\titleformat*{\subsection}{\large\bfseries}
\titleformat*{\subsubsection}{\large}

% Head and foot edition
\usepackage{fancyhdr}
\pagestyle{fancy}
\fancyhf{}
\lhead{\small Mécanique des Fluides, Thermique et Énergétique}
\rhead{\small Théo \bsc{Delecour}}
\cfoot{\thepage}
\renewcommand{\headrulewidth}{0.2pt}
\renewcommand{\footrulewidth}{0.0pt}

\title{Mécanique des Fluides, Thermique et Énergétique}
\author{Théo Delecour}
\date{2015}

\begin{document}

% keeps title, author and date attributes
\makeatletter

% ================================================================
\begin{center}
 \LARGE \@title
 
 \Large \@author
 
 \large \href{mailto:theo.delecour@ecl13.ec-lyon.net}{theo.delecour@ecl13.ec-lyon.fr}
\end{center}

\makeatother

Ce document résume des éléments utiles pour l'étude de la mécanique des fluides, depuis les principes généraux qui entrent en jeu jusqu'aux applications concrètes des équations. Ce document se veut général et non spécialisé.

% ----------------------------------------------------------------
\printnomenclature

% EoS
\nomenclature{$p$}{pression statique (Pa)}
\nomenclature{$T$}{température (K)}
\nomenclature{$\rho$}{masse volumique (kg/m\up{3})}
\nomenclature{$V$}{volume (m\up{3})}

\nomenclature{$U$}{vitesse du fluide (m/s)}

% Thermo
\nomenclature{$e$}{énergie interne spécifique (J/kg)}
\nomenclature{$h$}{enthalpie spécifique (J/kg)}
\nomenclature{$\gamma$}{coefficient de Laplace du gaz parfait (sans unité)}

% Conservation
\nomenclature{$z$}{hauteur (m)}
\nomenclature{$g$}{gravité terrestre (m/s\up{2})}
\nomenclature{$\vec{\vec{\sigma}}$}{tenseur des contraintes (Pa)}
\nomenclature{$\vec{\vec{I}}$}{matrice identité}

% Re, friction
\nomenclature{$l$}{longueur caractéristique (m)}
\nomenclature{$\mu$}{viscosité dynamique (Pa.s)}
\nomenclature{$\nu$}{viscosité cinématique (m\up{2}/s)}
\nomenclature{$f$}{coefficient de friction (sans unité)}
\nomenclature{$Re$}{nombre de Reynolds (sans unité)}
\nomenclature{$\epsilon$}{rugosité (m)}

% Thermique
\nomenclature{$c_v$}{capacité thermique spécifique à volume constant (J/K/kg)}
\nomenclature{$c_p$}{capacité thermique spécifique à pression constante (J/K/kg)}
\nomenclature{$k$}{conductivité thermique (W/m/K)}
\nomenclature{$\alpha$}{diffusivité thermique (m\up{2}/s)}
\nomenclature{$h_c$}{coefficient de convection (W/m\up{2}/K)}
\nomenclature{$\vec{\phi}$}{densité de flux de chaleur (W/m\up{2})}

% ----------------------------------------------------------------
\section*{Convention de notation}
Pour améliorer la visibilité des équations et la compréhension des problèmes, on utilisera l'opérateur $\vec{\nabla}$, pour rappel :
%
\begin{equation*}
\begin{aligned}
div (\vec{U}) &= \vec{\nabla} \cdot \vec{U} \\
\vec{grad} (\phi) &= \vec{\nabla} \phi \\
\Delta (\phi) &= \nabla^2 \phi
\end{aligned}
\end{equation*}



% ================================================================
\section{Éléments de cinématique}

En mécanique des fluides, on s'intéresse au comportement de \textbf{particules fluide} de petite taille (relativement aux dimensions caractéristiques de l'écoulement). Chacune est associée à un point matériel, et les grandeurs descriptives du fluide (vitesse $\vec{U}$, densité $\rho$...) ne varient pas au sein d'une même particule. Ces particules forment ensemble le fluide en mouvement, qui est alors considéré comme un \textbf{milieu continu}.

\paragraph{Lignes de courant :}ce sont des courbes qui décrivent un fluide en mouvement pour un temps $t$ fixé. En un point $M$ donné, la tangente à une ligne de courant donne la direction du vecteur vitesse de ce point (ou plutôt de la particule fluide associée au point).
%
\begin{equation}
\vec{U}\wedge d\vec{x}=0
\end{equation}

\paragraph{Trajectoire :}la trajectoire d'une particule fluide dépend de tout l'historique de l'écoulement :
%
\begin{equation}
\frac{d\vec{x}}{dt} = \vec{U}(\vec{x},t)
\end{equation}

Un écoulement \textbf{stationnaire} est un écoulement dont les propriétés ne dépendent pas du temps (on pourrait le qualifier de stable par exemple). Il ne faut cependant pas le confondre avec un écoulement statique où rien ne bouge (la statique des fluides est détaillée en section \ref{sec:statique}). Dans un écoulement stationnaire, les particules se déplacent le long des lignes de courant, on a alors coïncidence des lignes de courant et des trajectoires.

\paragraph{Dérivée matérielle :}elle représente la dérivée d'une grandeur $\phi(\vec{x},t)$ par rapport au temps en prenant en compte le mouvement du fluide avec le temps :
%
\begin{equation}
\frac{D\phi}{Dt} = \frac{\partial{\phi}}{\partial{t}} + \left( \vec{U} \cdot \vec{\nabla} \right) \phi
\end{equation}
%
Le terme $\left( \vec{U} \cdot \vec{\nabla} \right) \phi$ est le terme advectif, lié au mouvement du fluide, on appelle même couramment $\left( \vec{U} \cdot \vec{\nabla} \right)$ opérateur advection (voir sur \href{https://fr.wikipedia.org/wiki/Advection}{Wikipédia}).

\paragraph{Théorème de la divergence :}c'est un théorème basique et important, il décrit la correspondance entre le flux d'une quantité entrant dans un volume par sa frontière (sa surface) et la divergence de cette quantité dans le volume :
%
\begin{equation}
\int_{S} {\vec{U}.\vec{n}~dS} = \int_{V} { \vec{\nabla} \cdot \vec{U} dV }
\end{equation}

\paragraph{Théorème de Reynolds :}ou théorème de transport, il exprime le dérivée temporelle d'une quantité $\Phi$ qui possède un pendant volumique $\phi$ (par exemple une masse $m$ et une masse volumique $\rho$) :
%
\begin{equation}
\frac{d\Phi}{dt}
 = \frac{d}{dt} \int_{V(t)} \phi~dV
 = \int_{V(t)} \frac{\partial \phi}{\partial t} dV
 + \int_{S(t)} \phi \vec{U}.\vec{n} dS
\end{equation}

% ================================================================
\section{Statique de fluides}\label{sec:statique} %OK, une figure peut etre

Le fluide est ici au repos, on s'intéresse alors particulièrement l'évolution de sa pression $p$.


%-------------------------------------------------------
\subsection{Équation d'état}
\paragraph{Gaz parfait :}relation très célèbre, bien vérifiée à faible pression (p>1 atm) et température supérieures à 273K
%
\begin{equation}
pV = nRT \text{ avec } R = k_B \times \mathcal{N}_A = 8.314 \text{ J/K/mol}
\end{equation}
%
avec $k_B$ la constante de Boltzmann et $\mathcal{N}_A$ la constante d'Avogadro. On définit souvent $r = R/M$ avec $M$ la masse molaire du gaz étudié (en kg/mol), on a ainsi \textbf{pour l'air} :
%
\begin{equation}
p = \rho~r~T \quad\text{avec } r = 287 \text{ J/K/kg}
\end{equation}

\paragraph{Gaz de Van der Waals :}meilleure approximation que l'équation des gaz parfait pour des hautes pressions et des faibles températures. Ce modèle prend en compte le volume propre des molécules et les forces d'interactions de Van der Waals ($a$ et $b$ sont des constantes propres au gaz étudié) :
%
\begin{equation}
\left( p + a\rho^2 \right) \left( \frac{1}{\rho} - b \right) = rT
\end{equation}


%-------------------------------------------------------
\subsection{Hydrostatique}
On démontre la relation de l'hydrostatique (ou Principe Fondamental de la Statique) :
%
\begin{equation}
\vec{\nabla} p = \rho \vec{g}
\label{eq:PFS}
\end{equation}

L'application de cette relation à une atmosphère uniquement composée d'un gaz parfait isotherme nous donne la variation de la pression avec l'altitude :
%
\begin{equation}
    p = p_0~\exp{\left( -\frac{g}{rT}z \right)}
\end{equation}

On peut aussi démontrer des relations similaires pour un fluide en connaissant à son coefficient de compressibilité isotherme $\upchi_T$ ou un gaz polytropique.

\subsection{Loi de Jurin}
Un fluide est placé dans un tube au repos, il présente une surface de contact avec l'air. La loi de Jurin permet de lier l'angle que forme le ménisque du liquide à la différence de pression entre le liquide et l'air. On peut également calculer la hauteur de monter du liquide dans un capillaire. Les phénomènes de mouillabilité du contact fluide-solide (le solide étant le tube ici) entrent en jeu.

\subsection{Principe d'Archimède}
\paragraph{Enoncé :}Dans une situation d'équilibre, tout corps partiellement immergé entre deux fluides au repos subit une poussée verticale ascendante qui est égale aux poids des volumes des deux fluides déplacés.

Cette poussée est appliquée au centre de masse du volume de fluide déplacé et non au centre de masse du solide (point où s'applique la gravité). Apparaît ainsi la problématique d'équilibre du solide en fonction du placement relatif de ces points.


% ================================================================
\section{Thermodynamique}
L'enthalpie spécifique est définie ainsi :
%
\begin{equation}
h = e + \frac{p}{\rho}
\end{equation}

Pour un gaz parfait, on a :
%
\begin{align}[left=\empheqlbrace]
 &de = c_v dT\\
 &dh = c_p dT
\end{align}
%
Cette relation d'intègre directement si $c_v$ et $c_p$ sont indépendants de $T$. Pour un gaz parfait, la relation de Mayer $c_p - c_v = r$ et le fait que $c_p/c_v = \gamma$ nous donnent :
%
\begin{align}[left=\empheqlbrace]
 &c_p = \frac{\gamma r}{\gamma-1}\\
 &c_v = \frac{\gamma}{\gamma-1}
\end{align}
%
Le coefficient $\gamma$ vaut 5/3 pour un gaz parfait monoatomique et 7/5 pour un diatomique. Ainsi, pour un gaz parfait, $c_v$ et $c_p$ sont théoriquement indépendants de $T$, cependant les valeurs réelles peuvent en dépendre.

% ------------------------------------------------------
\subsection{Premier principe}\label{subsec:1erprincipe}



% ------------------------------------------------------
\subsection{Second principe}



L'application des principes de la thermodynamique à une transformation \textbf{isentropique} (ou \textbf{adiabatique et réservible}) nous donne les équations de Laplace, toutes liées par la relation des gaz parfaits et dont voici la plus courante :
%
\begin{equation}
\frac{p}{\rho^{\gamma}} = Cst
\end{equation}


% ================================================================
\section{Viscosité et rhéologie}\label{sec:rheo}
Le caractère visqueux d'un fluide permet d'assurer la continuité de son profil de vitesse, en particulier à proximité des parois. En quelque sorte, c'est grâce à sa viscosité qu'un fluide peut avoir une vitesse nulle à une paroi immobile mais garder une vitesse moyenne élevée.

La rhéologie d'un fluide décrit son comportement visqueux. Un modèle rhéologique donne la relation liant la contrainte de cisaillement $\tau$ (en $Pa$) d'un fluide à son taux de cisaillement $\dot{\gamma}$ (en $s^{-1}$), il en existe plusieurs couramment utilisés.

Le modèle Newtonien est le plus simple, l'eau vérifie ce modèle avec fidélité. Il respecte la loi de viscosité de Newton qui fait apparaître la viscosité dynamique $\mu$ du fluide :
%
\begin{equation}
\tau = \mu \dot{\gamma} = \mu \left( \frac{\partial U}{\partial x} \right)
\end{equation}

La viscosité $\mu$ d'un fluide dépend fortement de sa température ; en général la viscosité d'un liquide diminue avec $T$ tandis que celle d'un gaz augmente.

A retenir : un fluide à viscosité extrêmement élevée a un comportement proche de celui d'un solide, on pourrait même considérer les solides comme des fluides à viscosité quasi-infinie.

% ================================================================
\section{Lois de conservation}

% ------------------------------------------------------
\subsection{Forme générale}



Il existe 3 lois de conservation essentielles, conservation de :
\begin{itemize}\renewcommand{\labelitemi}{$\bullet$}
\item la masse
\item la quantité de mouvement (ou moment)
\item énergie totale
\end{itemize}

% ------------------------------------------------------
\subsection{Conservation de la masse}
De façon très générale, il faut toujours se souvenir que la masse se conserve. Un système fermé peut voir son volume, sa pression, sa température changer, mais pas sa masse (sauf si on rajoute ou on enlève de la matière, mais ce sont des cas très particuliers). Cette notion est comme une vérité ultime en cas de doute.
%
\begin{equation}
\frac{dm}{dt} = \frac{d}{dt} \int_{V(t)} {\rho~dV} = 0
\end{equation}
%
L'application du théorème de Reynolds nous donne sa forme la plus courante qu'on nomme \textit{équation de la masse} ou \textit{équation de continuité} :
%
\begin{equation}
\frac {D\rho}{Dt} + \rho \nabla \cdot \vec{U} = 0
\label{eq:masse}
\end{equation}
%
\paragraph{Compressibilité du fluide :} la masse d'une quantité donnée de fluide se conserve, cependant la densité $\rho$ de ce fluide n'est pas obligatoirement constante. Cela signifie que le volume de ce fluide peut varier : il est compressible. Un fluide incompressible a sa densité $\rho$ qui est constante : on ne peut pas le comprimer, comme un solide parfait ; son volume se conserve. L'équation de conservation de la masse nous donne alors une relation utile :
%
\begin{equation}
\frac {D\rho}{Dt} = 0 \Rightarrow \nabla \cdot \vec{U} = 0
\end{equation}

% ------------------------------------------------------
\subsection{Conservation de la quantité de mouvement}
Cette loi correspond au principe fondamental de la dynamique appliqué à une particule fluide. Son équation est très semblable à la fameuse $m\vec{a}=\sum \vec{F}_{ext}$
%
\begin{equation}
\rho \frac{D\vec{U}}{Dt} = \vec{\nabla} \cdot \vec{\vec{\sigma}} + \rho \vec{f}
\label{eq:QDM}
\end{equation}
%
Le terme à gauche représente l'inertie du fluide : ($\frac{D\vec{U}}{Dt}$) est l'accélération

\begin{itemize}\renewcommand{\labelitemi}{$\bullet$}
\item $\vec{\vec{\sigma}}$ est le tenseur des contraintes (en Pa), il représente les forces qui agissent sur le fluide \textit{en surface}, tout ce qui est pression et frottement est englobé par ce terme

\item $\vec{f}$ représente les efforts qui agissent \textit{en volume} (en N/m\up{3}), les principaux efforts qui rentrent dans ce cas sont la gravité terrestre ($\vec{g}$) et la gravitation universelle
\end{itemize}

On peut ainsi calculer à un instant donné l'ensemble des forces agissant sur un domaine $V$ de surface $S$ :
%
\begin{equation}
\vec{F} = \int_V { \left( \vec{\nabla} \cdot \vec{\vec{\sigma}} + \rho \vec{f} \right) dV }
        = \int_S { \vec{\vec{\sigma}}.\vec{n} dS} + \int_V {\rho \vec{f} dV }
\end{equation}

De façon générale, $\vec{\vec{\sigma}}.\vec{n} = \vec{T}$ représente les efforts exercés sur une surface de normale $\vec{n}$

% ================================================================
\section{Fluide newtownien}

Le tenseur des contraintes $\vec{\vec{\sigma}}$ s'exprime ainsi :
%
\begin{equation}
\vec{\vec{\sigma}} = - p \vec{\vec{I}} + \vec{\vec{\tau}}
\end{equation}
%
avec $\vec{\vec{I}}$ la matrice identité et $\vec{\vec{\tau}}$ le tenseur des contraintes visqueuses.

L'injection de cette définition dans la conservation de la QDM \eqref{eq:QDM} nous donne :
%
\begin{equation}
\rho \frac{D\vec{U}}{Dt} = - \vec{\nabla} p + \vec{\nabla} \cdot \vec{\vec{\tau}} + \rho \vec{f}
\end{equation}

En statique (fluide sans mouvement), on a $\vec{\vec{\tau}} = 0$ et $\vec{U} = \vec{0}$, cette équation nous donne alors le Principe Fondamental de la Statique \eqref{eq:PFS}.

%-------------------------------------------------------
\paragraph{Loi newtonienne :}Elle est valable pour un fluide newtonien (voir partie \ref{sec:rheo} sur la rhéologie).
%
\begin{equation}
\vec{\vec{\tau}} = - \lambda \vec{\nabla} \cdot \vec{U} \vec{\vec{I}} + 2\mu \vec{\vec{D}}
\end{equation}
%
$\vec{\vec{D}}$ est le tenseur des vitesses de déformation (en s\up{-1}), il traduit la déformation des particules fluides avec le mouvement, on calcule ses termes ainsi :
%
\begin{equation}
D_{ij} = \frac{1}{2} \left( \frac{\partial{U_i}}{\partial{x_j}} + \frac{\partial{U_j}}{\partial{x_i}} \right)
\end{equation}

$\mu$ est la viscosité dynamique vur en partie \ref{sec:rheo} et $\lambda$ est le coefficient de seconde viscosité. Dans un cas incompressible, le fait que $\vec{\nabla} \cdot \vec{U} = 0$ implique que $\vec{\vec{\tau}} = 2 \mu \vec{\vec{D}}$. Dans un cas non visqueux on a simplement $\vec{\vec{\tau}} = 0$ (c'est un modèle très simplifié bien sûr).

La viscosité cinématique d'un fluide newtonien s'exprime en m/s\up{2} et est définit ainsi :
%
\begin{equation}
\nu = \frac{\mu}{\rho}
\end{equation}

%-------------------------------------------------------
\paragraph{Théorème de Navier Stokes :}il est simplement composé des équations de conservation de la masse \eqref{eq:masse} et de la QDM \eqref{eq:QDM} dans le cas d'un fluide newtonien, incompressible et homogène. On note qu'on a alors $\vec{\nabla} \cdot \vec{\vec{\tau}} = 2 \mu \vec{\nabla} \cdot \vec{\vec{D}} = \mu \nabla^2 \vec{U}$.
%
\begin{align}[left=\empheqlbrace]
 & \frac{D\vec{U}}{Dt} = - \frac{1}{\rho} \vec{\nabla} p + \nu  \nabla^2 \vec{U} + \vec{f} \\
 \notag & \vec{\nabla} \cdot \vec{U} = 0
\label{eq:navierstokes}
\end{align}


%-------------------------------------------------------
\paragraph{Théorème de Bernouilli :}ses conditions d'application sont les suivantes :

\begin{itemize}
    \item fluide non visqueux
    $\Leftrightarrow \vec{\vec{\sigma}}=-p\vec{\vec{I}}$
    \item écoulement stationnaire
    $\Leftrightarrow \partial{}/\partial{t} = 0$
    \item écoulement incompressible
    $\Leftrightarrow \vec{\nabla} \cdot \vec{U} = 0$
    \item forces massiques dérivant d'un potentiel
    $\Leftrightarrow\vec{f} = - \vec{\nabla} \psi$ (pour la pesenteur $\vec{f} = \vec{g}$ et $\psi = g z$)
\end{itemize}
%
Alors la quantité suivante est constante le long d'une ligne de courant :
%
\begin{equation}
K = p + \frac{1}{2}\rho U^2 + \rho\psi
\label{eq:bernouilli}
\end{equation}

\paragraph{Équations d'Euler :}elles correspondent au théorème de Navier Stokes dans le cas d'un fluide non visqueux ($\nu=0$)

\textbf{Conditions d'adhérence aux limites :}
\begin{itemize}
    \item paroi solide : $\vec{U}_{paroi}=$vitesse de la paroi
    \item interface fluides : $\vec{U}_A=\vec{U}_B$
\end{itemize}

\begin{center}
    \begin{tikzpicture}
        \fill[color=gray!20] (0,0) -- (0,2) ..controls +(2,0) and +(-2,0).. (5,0) -- (0,0);
        \draw (0,2) ..controls +(2,0) and +(-2,0).. (5,0);
        \draw (0.8,0.5) node[above] {$B$};
        \draw (4.7,2) node[below] {$A$};
        \draw [->,red,very thick] (2.5,1) -- +(1.4,0);
        \draw [red] (3.2,1) node[above] {$\vec{U}_A = \vec{U}_B$};
    \end{tikzpicture}
\end{center}


% ================================================================
\section{Nombre de Reynolds} %OK

% ------------------------------------------------------
\subsection{Définition}
Le nombre de Reynolds est un nombre sans dimension fondamental en mécanique des fluides, il représente l'importance relative des forces d'inertie sur les forces visqueuses de l'écoulement. Pour un fluide newtonien, on le calcule ainsi :
%
\begin{equation}
    Re = \frac{Ul}{\nu}
       = \frac{\rho U l}{\mu}
\end{equation}
%
avec $l$ longueur caractéristique de l'écoulement (voir partie \ref{sec:Re_details} pour plus de détails). On raisonne généralement en ordre de grandeur et non en valeur exacte de $Re$.

Le nombre de Reynolds (1883) permet de définir le \textbf{régime} d'un écoulement. On distingue 2 grands régimes :
%
\begin{itemize}
    \item le régime \textbf{laminaire} à $Re<10^3$ environ. Les forces visqueuses sont importantes, l'écoulement est régulier et ordonné, les lignes de courant sont bien définies. C'est le cas d'un écoulement lent d'un fluide très visqueux (de l'huile ou du miel par exemple)

    \item le régime \textbf{turbulent} à $Re>10^5$ environ. L'écoulement devient chaotique, ses caractéristiques (la vitesse par exemple) prennent des valeurs aléatoires dans des directions aléatoires, mais centrées autour d'un champ moyen. C'est le cas d'un écoulement rapide d'un fluide peu visqueux (de l'eau ou de l'air)
\end{itemize}


%-------------------------------------------------------
\subsection{Faibles nombres de Reynolds}
Pour $Re \ll 1$, le caractère laminaire du régime est poussé à l'extrême, aucune turbulence n'est considérée, on entre en \textit{régime de Stokes}. Il est régi par les équations suivantes, elles sont dérivées des équations de Navier-Stokes, avec suppression des termes inertiels (car $Re \ll 1$) :
%
\begin{align}[left=\empheqlbrace]
    & \mu \nabla^2 \vec{U} = \vec{\nabla} p\\
    \notag & \vec{\nabla} \cdot \vec{U} = 0
\end{align}
%
Les équations différentielles sont linéaires ici, ce qui rend le problème plus simple à résoudre qu'un cas à $Re$ élevé. Les écoulements concernés sont de type microscopique (par exemple dans des capillaires) ou, à l'opposé, de type géologiques (par exemple le mouvement des glaciers).


%-------------------------------------------------------
\subsection{Régime turbulent}
Un écoulement turbulent autour d'un objet fait apparaître un \textit{sillage} derrière celui-ci, ainsi qu'une \textit{couche} limite autour de ses parois. La vitesse de l'écoulement est plus faible dans le sillage et la couche limite, les effets de la viscosité sont plus importants et non négligeables ici.

On appelle souvent le couche limite \textit{couche visqueuse}, c'est en effet au sein de celle-ci que la majorité des effets visqueux ont lieu. On considère parfois que l'écoulement en couche limite est laminaire, cependant à $Re$ très élevé, la couche limite devient elle-même turbulente.

Les équations de Navier-Stokes à $Re \gg 1$ nous donnent les ordres de grandeur suivants :
%
\begin{align}
    & \frac {\delta}{l} = O \left( \frac{1}{\sqrt{Re}} \right) \\
    & p = O \left( \rho U^2 \right)
\end{align}
%
On lit sur la première équation que pour $Re \to \infty$, la couche limite s'amincit. Le modèle non visqueux peut être appliqué en dehors de la couche limite. La seconde équation permet d'estimer les variations de pression en régime turbulent, utile pour un calcul rapide.

\paragraph{Équations de Prandtl :}elles décrivent l'écoulement à l'intérieur de la couche limite :

Même pression dans la couche limite et dans le reste de l'écoulement.


%-------------------------------------------------------
\subsection{Détails}\label{sec:Re_details}
La définition de la longueur caractéristique $l$ d'un écoulement peut sembler vague, mais à nouveau on raisonne surtout en ordre de grandeur pour le calcul de $Re$, donc pas besoin d'une valeur très précise pour $l$. Pour un écoulement autour d'un objet (dit \textit{externe}), on prendra la taille de cet objet dans la direction de l'écoulement (par exemple la longueur d'une voiture sur l'autoroute ou la longueur d'un profil d'aile) :
%
\begin{center}
    \begin{tikzpicture}[scale=0.6]
        \draw (0,0) circle(1);
        \draw [<->] (-1,1.2) -- (1,1.2);
        \draw (0,1.2) node[above] {$l$};
        \draw (4,0) ..controls +(0,0.5) and +(-1,0).. (7,0);
        \draw (4,0) ..controls +(0,-0.5) and +(-1,0).. (7,0);
        \draw [<->] (4,0.5) -- (7,0.5);
        \draw (5.5,0.5) node[above] {$l$};
    \end{tikzpicture}
\end{center}
%
Pour un écoulement dans une conduite (dit \textit{interne}), on utilise souvent le \textit{diamètre hydraulique}. Il est défini par $D_h = 4 \cdot S/P$ avec $S$ la surface transverse à l'écoulement et $P$ le périmètre de cette surface. Pour un tuyau, il correspond à son diamètre (logique). Pour une conduite en anneau, il vaut la différence des diamètres extérieur et intérieur de l'anneau.

Pour un écoulement dans une conduite circulaire, les nombres de Reynolds limites ont été précisément identifiés par expérience :
\begin{itemize}
    \item régime laminaire : $Re<2100$
    \item régime turbulent : $Re>4000$
\end{itemize}


% ================================================================
\section{Friction et perte de charge} %OK
La perte de charge est une perte de pression que subit un fluide qui circule dans une conduite. Cette perte se produit par friction du fluide aux parois (frottement) ou par changements brutaux de section ou de direction de l'écoulement. Ces pertes sont particulièrement importantes dans l'étude des réseaux hydrauliques.

Toute perte de charge peut se modéliser par une perte de "hauteur" du fluide. En effet à toute pression $p$ en $Pa$ correspond une hauteur en mètres par la relation suivante :
%
\begin{equation}
\Delta z = \frac{p}{\rho g}
\end{equation}
%
$h$ représente une hauteur fictive atteinte si toute l'énergie de pression du fluide était convertie en énergie potentielle.
%
\begin{center}
  \begin{tikzpicture}[scale=0.6]
  \draw (0,0) circle(1);
  \draw (0,1) -- (4,1);
  \draw (0,-1) -- (4,-1);
  \draw (4,1) arc(90:-90:1); %arc(angle de départ:angle d'arrivée:rayon)
  %L
  \draw [<->] (0,-1.2) -- (4,-1.2);
  \draw (2,-1.2) node[below] {$L$};
  %D
  \draw [<->] (0,0.9) -- (0,-0.9);
  \draw (0,0) node[right] {$D$};
  %U
  \draw [magenta,->,very thick] (-1.5,0) -- (-0.5,0);
  \draw [magenta] (-1.2,0) node[above] {$\vec U$};
  \end{tikzpicture}
\end{center}
%
Pour un écoulement incompressible d'un fluide newtonien en régime \textbf{laminaire} dans une conduite de diamètre $D$ et de longueur $\Delta L$, on peut montrer analytiquement que :
%
\begin{equation}
\Delta p_f = \frac{2\rho U^2}{D} f \Delta L \quad\text{avec } f = \frac{16}{Re}
\end{equation}

$f$ est le coefficient de friction de Fanning. L'expression de $\Delta p_f$ en fonction de $f$ reste valable en dehors des hypothèses précédentes, elle est donc valable pour des fluides non newtoniens et quelque soit le régime d'écoulement. Ce qui changera pour ces cas sera la définition du coefficient de friction $f$. On notera aussi que pour des conduites en anneaux, la définition de $\Delta p_f$ peut être reprise avec $D = D_h$ diamètre hydraulique.

Il existe une autre description du coefficient de friction : celle de Darcy (1856) (ou Moody dans certaines publications), la relation entre coefficient de Darcy et Fanning est très simple :
%
\begin{equation}
f_{\text{Darcy}} = 4\cdot f_{\text{Fanning}}
\end{equation}

Pour un régime \textbf{turbulent}, le coefficient de friction de Fanning vérifie l'équation suivante dite de Colebrook (1939) :
%
\begin{equation}
\frac{1}{\sqrt{f}} = -4\cdot\log\left(0.269~\frac{\epsilon}{d} + \frac{1.255}{Re\sqrt{f}}\right)
\end{equation}
%
Cette formulation est implicite ($f$ se situe des 2 côtés de l'équation) et fait intervenir la rugosité de la paroi de la conduite $\epsilon$ (en $m$). Cette équation se résout avec une méthode itérative (type Newton-Raphson). De nombreuses écritures explicites (donc plus simples mais approximatives) du coefficient de friction ont été formulées. On peut les trouver sur \href{https://en.wikipedia.org/wiki/Darcy_friction_factor_formulae}{Wikipédia}, la plus connue est celle de Blasius (1913) :
%
\begin{equation}
f = \frac{0.0791}{Re^{0.25}}
\end{equation}

% ================================================================
\section{La turbulence}
Vorticité $\vec{\omega}=\vec{rot}(\vec{U})$

Irrotationnalité : $\vec{\omega}=\vec{0}$


% ================================================================
\section{Écoulements compressibles}

Validité du modèle fluide incompressible : $Ma \ll 1$ 
avec $Ma=\frac{U}{c}$ et $c=\sqrt{\gamma r T}$


% ================================================================
\section{Transfert de chaleur}

\paragraph{Conductivité thermique :}on la note $k$ ou $\lambda$ et elle s'exprime en W/K/m. Elle caractérise la capacité d'un matériau à conduire la chaleur (plus elle est élevée, plus le matériau conduit rapidement). Elle est assez liée à la conductivité électrique $\sigma$ du matériau (un matériau bon conducteur de chaleur est souvent également bon conducteur d'électricité), pour un \textbf{métal} la loi de Wiedmann-Franz affirme que :
%
\begin{equation}
    \frac{k}{\sigma~T} = L_0 = 2.44~10^{-8} W\Omega K^{-2}
\end{equation}
%
On parle aussi bien de problème de conduction que de diffusion, ici ces 2 termes sont identiques.


\paragraph{Loi de Fourier :}c'est la loi essentielle dans tout problème de diffusion (de la chaleur ou de la matière par exemple). Elle s'applique à tout matériau isotrope (une formulation plus complexe reste possible pour un matériau non isotrope). En thermique, elle représente la relation entre la densité de flux de chaleur $\vec{\phi}$ traversant un volume et le gradient de température au sein de ce volume :
%
\begin{equation}
    \vec{\phi} = - k \vec{\nabla}(T)
    \label{eq:fourier}
\end{equation}
%
La présence du signe négatif est physiquement logique : le flux va du chaud vers le froid alors que le gradient de température suit le sens inverse (il va des températures faibles aux températures élevées).
%
\begin{center}
    \begin{tikzpicture}[scale=0.7]
        \filldraw[fill=red!20,draw=red] (0,0) circle(1);
        \filldraw[fill=blue!20,draw=blue] (5,0) circle(1);
        \draw [red] (0,1) node[above] {chaud};
        \draw [blue] (5,1) node[above] {froid};
        % arrows
        \draw [magenta,->,very thick] (1.5,0.5) -- (3.5,0.5);
        \draw [->,very thick] (3.5,-0.5) -- (1.5,-0.5);
        % text
        \draw [magenta] (2.5,0.5) node[above] {$\vec{\phi}$};
        \draw (2.5,-0.5) node[below] {$T\nearrow$};
  \end{tikzpicture}
\end{center}


\paragraph{Diffusivité thermique :}c'est un rapport qu'on rencontre régulièrement en thermique, il caractérise également la capacité d'un matériau à diffuser la chaleur (présence importante de la conductivité) :
%
\begin{equation}
    \alpha = \frac{k}{\rho c_p}
\end{equation}


\paragraph{Équation de la température :}
%
\begin{equation}
    \rho~c_p \frac{DT}{Dt}
    = \vec{\nabla} \cdot \vec{\phi} %k \nabla^2 T
    + \rho~q_*
    + \vec{\vec{\tau}} : \vec{\vec{D}}
    + \beta T\frac{Dp}{Dt}
\end{equation}
%
On note que d'après \eqref{eq:fourier} on a : $\vec{\nabla} \cdot \vec{\phi} = - k \nabla^2 T$.

La géométrie du problème et les hypothèses physiques nous mènent souvent à adopter une forme simplifiée de cette équation. Dans un cas sans mouvement de fluide (conduction pure) et sans apport de chaleur, on a :
%
\begin{equation}
    \frac{\partial T}{\partial t}
    = \alpha \nabla^2 T
\end{equation}

Un élément utile à retenir : la diffusion de chaleur tend à lisser le champ de température, à diminuer les écarts de température. Les pics de chaleur initiaux sont étalés en distance avec le temps, et cette distance à l'instant $t$ est de l'ordre de $\sqrt{\alpha~t}$ (en m).


\paragraph{Nombre de Péclet :}il représente le rapport entre effets d'inertie et effets thermiques, c'est le pendant du nombre de Reynolds dans le domaine de la thermique
%
\begin{equation}
    Pe = \frac{Ul}{\alpha}
\end{equation}
%
On observe aussi l'existence de couches et sillages thermiques autour des obstacles à $Pe$ élevé (attention ces couches sont à distinguer des couches limites visqueuses).

\paragraph{Nombre de Prandt :}il est définit comme le rapport des diffusivités visqueuse et thermique, il représente ainsi l'importance relative entre effets visqueux et thermiques.
%
\begin{equation}
    Pr = \frac{\nu}{\alpha}
       = \frac{\mu~c_p}{k}
       = \frac{Pe}{Re}
\end{equation}
%
Il est très élevé ($10^3$) dans le cas d'une huile (grande viscosité) et très faible ($10^{-2}$) pour un métal liquide (excellents conducteurs thermiques et peu visqueux). Pour un gaz, il est d'environ $0.7$, et pour l'eau il est compris entre $1$ et $10$.

%\begin{center}
%  \begin{tikzpicture}
%  \fill[fill=gray!20] (-4,2) -- (0,2) -- (0,-1) -- (-4,-1);
%  \fill[fill=gray!40] (0,2) -- (4,2) -- (4,-1) -- (0,-1);
%  \draw [very thick,->] (-4,0) -- (4,0);
%  \draw (0,-1) -- (0,2);
%  \draw (0.5,0) node[below] {$x=0$};
%  \draw (-2,1) node[above] {$T_A(x)$};
%  \draw (2,1) node[above] {$T_B(x)$};
%  \end{tikzpicture}
%\end{center}


% ------------------------------------------------------
\subsection{Flux de conduction et convection}
La convection est le second grand phénomène de la thermique, après la conduction. Elle se définit comme un transfert d'énergie directement lié à un transport de masse. Ainsi, dès qu'il y a mouvement, il y a convection, le cas le plus courant est celui d'un fluide en mouvement par rapport à un solide (on sera dans ce cadre ici). On distingue 2 types de convection : la forcée et la libre (ou naturelle), décrites ci-dessous. Le coefficient de convection $h$ est dans tous les cas un nombre important dans la prédiction des comportements convectifs.

Ce nombre $h$ intervient dans l'écriture des flux de chaleur aux parois. Ces flux aux parois peuvent définir les conditions limites des problèmes thermiques. Pour une paroi plane située à l'abscisse $x_0$ et séparant les milieux 1 et 2, les densité de flux s'écrivent ainsi :
%
\begin{align}[left=\empheqlbrace]
    &\phi_{\text{conduction}}
    = - k_1 \left. \frac{\partial{T_1}}{\partial{x}} \right|_{x=x_0}
    = - k_2 \left. \frac{\partial{T_2}}{\partial{x}} \right|_{x=x_0} \\
    &\phi_{\text{convection}}
    = h_{12} \left( T_2 - T_1 \right)_{x=x_0}
    \label{eq:fluxth}
\end{align}


% ------------------------------------------------------
\subsection{Convection forcée :}
Le phénomène de convection forcée apparaît lorsque le fluide est mis en mouvement par
l'écoulement du fluide est , par une pompe par exemple. Ce mouvement forcé du fluide et à l'origine d'échange convectifs avec le solide en contact. Cependant, au sein même du fluide, son mouvement fait déjà apparaître un terme advectif dans son équation de la chaleur. Ce terme est dû à la dépendance de $T$ en temps, et témoigne du transport de la chaleur avec l'écoulement à la vitesse $\vec{U}$ :
%
\begin{equation}
    \rho~c_{p} \left( \frac{\partial T}{\partial t}
    + \vec{U} \cdot \vec{\nabla} T \right)
    = - \vec{\nabla} \cdot \vec{\phi}
\end{equation}

\paragraph{Nombre de Nusselt :}il est défini comme le rapport entre le gradient de température à la paroi et le gradient de température "élargi". Il permet de calculer le coefficient de convection $h$.
%
\begin{equation}
    Nu = \frac{h~l}{k_f}
       = \frac{\left. \frac{\partial{T}}{\partial{x}} \right|_{x=x_0}}
              {\left( T_f - T_s \right)/l}
\end{equation}
%
avec $k_f$ le coefficient de conduction thermique du fluide en mouvement. Une analyse dimensionnelle (détaillée en \cite{battaglia2010introduction}) affirme que ce nombre peut être écrit en fonction des nombres de Reynolds $Re$ et Prandt $Pr$ (ou indifféremment Péclet $Pe$).

On retrouve l'écriture de $\phi_{\text{convection}}$ \eqref{eq:fluxth} ainsi :
%
\begin{equation}
    \phi = - k_f \left. \frac{\partial{T}}{\partial{x}} \right|_{x=x_0}
         = k_f \frac{Nu}{l} \left( T_s - T_f \right)
         = ...
\end{equation}

En conclusion, pour un volume $V$ de fluide en mouvement et en contact avec un solide sur une surface $S$, l'équation de la chaleur a la forme suivante :
%
\begin{equation}
    \rho~c_p \left( \frac{\partial T}{\partial t}
    + \vec{U} \cdot \vec{\nabla} T \right)
    = \frac{h S}{V} \left( T_s - T \right)
\end{equation}
%
$S$ est bien la surface de contact entre fluide et solide. Il existe de nombreuses prescriptions empiriques du nombre de Nusselt $Nu$ pour des convections forcées internes (fluide dans une conduite) ou externes (solide plongé dans un écoulement).


% ------------------------------------------------------
\subsection{Convection libre :}
Ici le fluide est initialement au repos, et c'est le transfert de chaleur qui le met en mouvement. Un gradient de température apparaît au sein du fluide, et la dépendance de sa densité $\rho$ avec la température est à l'origine d'un mouvement du fluide, par poussée d'Archimède (le fluide plus léger monte etc). Ce mouvement du fluide est dissipé par effets visqueux et par diffusion thermique (la diffusion tend à uniformiser la température, donc à réduire les différences de densité).

On peut également définir un nombre de Nusselt $Nu$, mais il ne sera plus exprimé en fonction de $Re$, car la vitesse $\vec{U}$ n'a pas de sens en convection libre (elle n'est pas nulle, mais il est impossible de vraiment la déterminer). Le nouveau coefficient important en convection libre est le coefficient de dilatation thermique $\beta$ du fluide (en K\up{-1}) :
%
\begin{equation}
    \beta = \frac{1}{V} \left( \frac{\partial V}{\partial T} \right)_p
          = - \frac{1}{\rho} \left( \frac{\partial \rho}{\partial T} \right)_p
\end{equation}
%
Ce coefficient vaut $\beta = 1/T$ pour un gaz parfait. Dans le cadre de notre étude, on suppose que les variations de densité $\rho$ avec la température $T$ sont faibles et linéaires en $T$ :
%
\begin{align}
    &\rho(T) = \rho_0(T_0) \left( 1 - \beta(T-T_0) \right) \\
    \notag &\beta (T-T_0) \ll 1
\end{align}

\paragraph{Nombre de Grashof :}il compare les forces visqueuses aux forces de gravité
%
\begin{equation}
    Gr = \frac{1}{\nu^2}\beta g l^3 \Delta T
\end{equation}
%
On peut exprimer $Nu$ en fonction de $Gr$ et $Pr$, ces relations sont déterminées expérimentalement et sont différentes selon la géométrie du problème.

\paragraph{Nombre de Rayleigh :}il est très lié à $Gr$ et permet de déterminer le régime d'écoulement en convection libre, on peut aussi déterminer $Nu = f(Pr,Ra)$
%
\begin{equation}
    Ra = Pr.Gr
       = \frac{1}{\nu\alpha}\beta g l^3 \Delta T
\end{equation}

Les prescriptions donnant le nombre de Nusselt $Nu$ sont encore plus rares et empiriques que dans le cas de la convection forcée. On remarque que la convection libre externe est plus aisée à étudier que la convection libre interne (où la fermeture du fluide implique beaucoup d'effets difficilement prévisibles). Également, le caractère chaud ou froid d'une source a son importance : comme le fluide monte généralement avec la chaleur, l'inversion des sources crée un tout nouveau problème.


% ------------------------------------------------------
\subsection{Convection mixte :}

En convection forcée et régime laminaire, il peut se créer dans certaines conditions un mouvement des particules dû à la convection naturelle (ce mouvement s'ajoute au mouvement initial des particules). Si le nombre de Prandtl $Pr$ est très faible ($10^{-2}$ ou moins), cela signifie que la couche thermique est plus épaisse que la couche visqueuse. Un gradient de température apparaît dans tout l'écoulement, et le mouvement des praticules est influencé par les variations de densité et les forces d'Archimède.

Il existe un critère plus simple pour déterminer l'importance relative de la convection forcée et de la convection libre sur une étude : le nombre adimensionné de Richardson :
%
\begin{equation}
    Ri = \frac{Gr}{Re^2}
\end{equation}
%
La convection est forcée pour $Ri \ll 1$ et naturelle pour $Ri \gg 1$.


% ================================================================
\section{Mélanges fluides} %OK
Fraction massique $Y_i$ :
%
\begin{equation}
Y_i = \frac{\rho_i}{\rho}
\end{equation}

\paragraph{Équation des espèces :} équation classique de diffusion (semblable à la diffusion de la température mais appliquée à la fraction massique de l'élément) :
%
\begin{equation}
\rho \frac{DY_i}{Dt} = R_i - \vec{\nabla} \cdot \vec{J_i}
\end{equation}
%
avec $R_i$ terme de réaction chimique, à l'origine d'une augmentation ou diminution de la fraction massique (analogue au terme $q_*$ de création de chaleur).

\paragraph{Loi de Fich :}similaire à la loi de Fourier, valable pour un mélange entre 2 fluides
%
\begin{equation}
\vec{J_i} = \rho~D~\vec{\nabla} Y_i
\end{equation}
%
avec $D$ la diffusivité

Cas sans réaction chimique : $R_i=0$
%
\begin{equation}
\frac{DY_i}{Dt} = D \nabla^2 Y_i
\end{equation}


% ================================================================
\section{Bonus vu en TD}
$Re$ grand $\Rightarrow$ variations de pression de l'ordre de $\rho U^2$, propriété 
tirée du Théorème de Bernouilli


% ================================================================
\nocite{*}
\bibliographystyle{plain}
\bibliography{biblio.bib}

\end{document}
