L'enthalpie spécifique est définie ainsi :
%
\begin{equation}
h = e + \frac{p}{\rho}
\end{equation}

Pour un gaz parfait, on a :
%
\begin{align}[left=\empheqlbrace]
 &de = c_v dT\\
 &dh = c_p dT
\end{align}
%
Cette relation d'intègre directement si $c_v$ et $c_p$ sont indépendants de $T$. Pour un gaz parfait, la relation de Mayer $c_p - c_v = r$ et le fait que $c_p/c_v = \gamma$ nous donnent :
%
\begin{align}[left=\empheqlbrace]
 &c_p = \frac{\gamma r}{\gamma-1}\\
 &c_v = \frac{\gamma}{\gamma-1}
\end{align}
%
Le coefficient $\gamma$ vaut 5/3 pour un gaz parfait monoatomique et 7/5 pour un diatomique. Ainsi, pour un gaz parfait, $c_v$ et $c_p$ sont théoriquement indépendants de $T$, cependant les valeurs réelles peuvent en dépendre.

% ------------------------------------------------------
\subsection{Premier principe}\label{subsec:1erprincipe}



% ------------------------------------------------------
\subsection{Second principe}



L'application des principes de la thermodynamique à une transformation \textbf{isentropique} (ou \textbf{adiabatique et réservible}) nous donne les équations de Laplace, toutes liées par la relation des gaz parfaits et dont voici la plus courante :
%
\begin{equation}
\frac{p}{\rho^{\gamma}} = Cst
\end{equation}
