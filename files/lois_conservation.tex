
% ------------------------------------------------------
\subsection{Forme générale}



Il existe 3 lois de conservation essentielles, conservation de :
\begin{itemize}\renewcommand{\labelitemi}{$\bullet$}
\item la masse
\item la quantité de mouvement (ou moment)
\item énergie totale
\end{itemize}

% ------------------------------------------------------
\subsection{Conservation de la masse}
De façon très générale, il faut toujours se souvenir que la masse se conserve. Un système fermé peut voir son volume, sa pression, sa température changer, mais pas sa masse (sauf si on rajoute ou on enlève de la matière, mais ce sont des cas très particuliers). Cette notion est comme une vérité ultime en cas de doute.
%
\begin{equation}
\frac{dm}{dt} = \frac{d}{dt} \int_{V(t)} {\rho~dV} = 0
\end{equation}
%
L'application du théorème de Reynolds nous donne sa forme la plus courante qu'on nomme \textit{équation de la masse} ou \textit{équation de continuité} :
%
\begin{equation}
\frac {D\rho}{Dt} + \rho \nabla \cdot \vec{U} = 0
\label{eq:masse}
\end{equation}
%
\paragraph{Compressibilité du fluide :} la masse d'une quantité donnée de fluide se conserve, cependant la densité $\rho$ de ce fluide n'est pas obligatoirement constante. Cela signifie que le volume de ce fluide peut varier : il est compressible. Un fluide incompressible a sa densité $\rho$ qui est constante : on ne peut pas le comprimer, comme un solide parfait ; son volume se conserve. L'équation de conservation de la masse nous donne alors une relation utile :
%
\begin{equation}
\frac {D\rho}{Dt} = 0 \Rightarrow \nabla \cdot \vec{U} = 0
\end{equation}

% ------------------------------------------------------
\subsection{Conservation de la quantité de mouvement}
Cette loi correspond au principe fondamental de la dynamique appliqué à une particule fluide. Son équation est très semblable à la fameuse $m\vec{a}=\sum \vec{F}_{ext}$
%
\begin{equation}
\rho \frac{D\vec{U}}{Dt} = \vec{\nabla} \cdot \vec{\vec{\sigma}} + \rho \vec{f}
\label{eq:QDM}
\end{equation}
%
Le terme à gauche représente l'inertie du fluide : ($\frac{D\vec{U}}{Dt}$) est l'accélération

\begin{itemize}\renewcommand{\labelitemi}{$\bullet$}
\item $\vec{\vec{\sigma}}$ est le tenseur des contraintes (en Pa), il représente les forces qui agissent sur le fluide \textit{en surface}, tout ce qui est pression et frottement est englobé par ce terme

\item $\vec{f}$ représente les efforts qui agissent \textit{en volume} (en N/m\up{3}), les principaux efforts qui rentrent dans ce cas sont la gravité terrestre ($\vec{g}$) et la gravitation universelle
\end{itemize}

On peut ainsi calculer à un instant donné l'ensemble des forces agissant sur un domaine $V$ de surface $S$ :
%
\begin{equation}
\vec{F} = \int_V { \left( \vec{\nabla} \cdot \vec{\vec{\sigma}} + \rho \vec{f} \right) dV }
        = \int_S { \vec{\vec{\sigma}}.\vec{n} dS} + \int_V {\rho \vec{f} dV }
\end{equation}

De façon générale, $\vec{\vec{\sigma}}.\vec{n} = \vec{T}$ représente les efforts exercés sur une surface de normale $\vec{n}$