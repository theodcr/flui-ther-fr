
De manière générale, pour tout fluide, le tenseur des contraintes $\vec{\vec{\sigma}}$ s'exprime ainsi :
%
\begin{equation}
    \vec{\vec{\sigma}} = - p \vec{\vec{I}} + \vec{\vec{\tau}}
\end{equation}
%
avec $\vec{\vec{I}}$ la matrice identité et $\vec{\vec{\tau}}$ le tenseur des contraintes visqueuses. Les contraintes qui agissent au sein d'un fluide se décomposent donc clairement en 2 types :
%
\begin{itemize}
    \item contraintes de pression (elles sont motrices de l'écoulement) avec $- p \vec{\vec{I}}$
    \item contraintes visqueuses (elles freinent l'écoulement) avec $\vec{\vec{\tau}}$
\end{itemize}

L'injection de cette définition dans la conservation de la QDM \eqref{eq:QDM} nous donne :
%
\begin{equation}
    \rho \frac{D\vec{U}}{Dt}
    = - \vec{\nabla} p
    + \vec{\nabla} \cdot \vec{\vec{\tau}}
    + \rho \vec{f}
\end{equation}

En statique (fluide sans mouvement), on a $\vec{U} = \vec{0}$ et par conséquent $\vec{\vec{\tau}} = 0$ (voir ci-dessous). Avec $\vec{f} = \vec{g}$ cette équation nous donne alors le Principe Fondamental de la Statique \eqref{eq:PFS}.

%-------------------------------------------------------
\paragraph{Loi newtonienne :}Elle est valable pour un fluide newtonien (voir partie \ref{sec:rheo} sur la rhéologie).
%
\begin{equation}
    \vec{\vec{\tau}} = 2\mu \vec{\vec{D}}
                       - \lambda \vec{\nabla} \cdot \vec{U} \vec{\vec{I}}
\end{equation}
%
\begin{itemize}
    \item $\vec{\vec{D}}$ est le tenseur des vitesses de déformation (en s\up{-1}), il traduit la déformation des particules fluides avec le mouvement, on calcule ses termes ainsi :
%
    \begin{equation}
        D_{ij} = \frac{1}{2} \left( \frac{\partial{U_i}}{\partial{x_j}}
                                  + \frac{\partial{U_j}}{\partial{x_i}} \right)
    \end{equation}

    \item $\mu$ est la viscosité dynamique vur en partie \ref{sec:rheo}
    \item $\lambda$ est le coefficient de seconde viscosité, l'hypothèse de Stokes assure que :
    \begin{equation}
        \lambda = - \frac{2}{3} \mu
    \end{equation}
\end{itemize}

Dans un cas incompressible, le fait que $\vec{\nabla} \cdot \vec{U} = 0$ implique que $\vec{\vec{\tau}} = 2 \mu \vec{\vec{D}}$. Dans un cas non visqueux on a simplement $\vec{\vec{\tau}} = 0$ (c'est un modèle très simplifié bien sûr).


La viscosité cinématique d'un fluide newtonien s'exprime en m/s\up{2} et est définit ainsi :
%
\begin{equation}
    \nu = \frac{\mu}{\rho}
\end{equation}

%-------------------------------------------------------
\paragraph{Théorème de Navier Stokes :}il est simplement composé des équations de conservation de la masse \eqref{eq:masse} et de la QDM \eqref{eq:QDM} dans le cas d'un fluide newtonien, incompressible et homogène. On note qu'on a alors $\vec{\nabla} \cdot \vec{\vec{\tau}} = 2 \mu \vec{\nabla} \cdot \vec{\vec{D}} = \mu \nabla^2 \vec{U}$, ce qui nous donne :
%
\begin{align}[left=\empheqlbrace]
    \label{eq:navierstokes}  % must put it before the first align equation
    & \rho \frac{D\vec{U}}{Dt} = - \vec{\nabla} p + \mu  \nabla^2 \vec{U} + \rho \vec{f} \\
    \notag & \vec{\nabla} \cdot \vec{U} = 0
\end{align}

On note qu'en écoulement incompressible, la valeur absolue de la pression $p$ n'importe pas, c'est uniquement son gradient $\vec{\nabla} p$ qui entre en jeu.

%-------------------------------------------------------
\paragraph{Théorème de Bernouilli :}il est très utile pour des calculs rapides, il faut bien retenir ses conditions d'application, qui sont les suivantes :

\begin{itemize}
    \item fluide non visqueux
    $\Leftrightarrow \vec{\vec{\sigma}}=-p\vec{\vec{I}}$
    \item écoulement stationnaire
    $\Leftrightarrow \partial{}/\partial{t} = 0$
    \item écoulement incompressible
    $\Leftrightarrow \vec{\nabla} \cdot \vec{U} = 0$
    \item forces massiques dérivant d'un potentiel
    $\Leftrightarrow\vec{f} = - \vec{\nabla} \psi$ (pour la pesenteur $\vec{f} = \vec{g}$ et $\psi = g z$)
\end{itemize}
%
Alors la quantité suivante est constante le long d'une ligne de courant :
%
\begin{equation}
    K = p + \frac{1}{2}\rho U^2 + \rho\psi
\label{eq:bernouilli}
\end{equation}

%-------------------------------------------------------
\paragraph{Équations d'Euler :}elles correspondent au théorème de Navier Stokes dans le cas d'un fluide non visqueux ($\nu=0$), c'est un cas très simple et rarement rencontré en réalisé.
%
\begin{align}[left=\empheqlbrace]
    & \frac{D\vec{U}}{Dt} = - \frac{1}{\rho} \vec{\nabla} p + \vec{f} \\
    \notag & \vec{\nabla} \cdot \vec{U} = 0
\end{align}

%-------------------------------------------------------
\paragraph{Conditions limites :}elle sont liées au problème donné et sont nécessaires à la résolution des équations. Une condition très importante est l'\textbf{adhérence} du fluide à son environnement en contact : la vitesse $\vec{U}$ du fluide est localement égale à la vitesse de la paroi ou de l'autre fluide en contact. Par exemple au niveau d'un contact entre le fluide et une paroi fixe selon le plan $x = 0$, on a $\vec{U}(x=0) = \vec{0}$.

\paragraph{Hypothèses courantes :}pour faciliter la résolution analytique ou numérique d'un problème, on émet des hypothèses souvent semblables aux suivantes :
%
\begin{itemize}
    \item vitesse unidirectionnelle (cas 1D type $\vec{U} = U_x \vec{e}_x$) ou bidirectionelle (cas 2D)
    \item pression uniforme dans la direction transversale de l'écoulement (cas 1D)
\end{itemize}

%\begin{center}
%    \begin{tikzpicture}
%        \fill[color=gray!20] (0,0) -- (0,2) ..controls +(2,0) and +(-2,0).. (5,0) -- (0,0);
%        \draw (0,2) ..controls +(2,0) and +(-2,0).. (5,0);
%        \draw (0.8,0.5) node[above] {$B$};
%        \draw (4.7,2) node[below] {$A$};
%        \draw [->,red,very thick] (2.5,1) -- +(1.4,0);
%        \draw [red] (3.2,1) node[above] {$\vec{U}_A = \vec{U}_B$};
%    \end{tikzpicture}
%\end{center}

